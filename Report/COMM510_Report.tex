%----- do not modify this first block,

\documentclass[sigconf,nonacm]{acmart}


\begin{document}
\title[Short title]{Which performance indicator is meaningful?}
\author{BART Number 208912}

%% Short summary of the report
\begin{abstract}
	
\end{abstract}

%% builds the first part of the formatted document.
\maketitle

\section{Introduction}
The use of Evolutionary Algorithms (EA) have been popular in solving multi-objective optimisation problems (MOP) due to their population-based approach that enables generation of several elements of the Pareto optimal set in a single run. These are particularly effective in solving complex MOPs with very large spaces, uncertainty, noise, disjoint Pareto curves, etc. Some examples of Multi-objective Evolutionary Algorithms (MOEA) are MOGA, NPGA, NSGA, NSGA-II, PAES, SPEA, SPEA2 and $\epsilon$-MOEA\cite{moea2007, deb2002}. Selecting an efficient MOEA towards solving MOPs of complex nature is crucial to yield accurate optimal points, especially when there is a computational cost attached to it. Performance Indicators (PI) are metrics that are helpful in comparing multiple MOEAs to determine their efficiencies. There are several PIs that exist in the literature in ranking MOEA. These are broadly divided as per cardinality, convergence, distribution, and both convergence and distribution. Their description is provided in section 2. This project explores the correlation between the run time and ranking of two MOEAs using different PIs.

\balance

\section{Background}
\subsection{Multi-objective Evolutionary Algorithms}
A multi-objective optimisation problem (MOP) \\
The two MOEAs considered in this study are NSGA-II and SPEA2.
\subsubsection{NSGA, NSGA-II, NSGA-III}
Non-dominated Sorting Genetic Algorithm
\subsubsection{SPEA, SPEA2}
\subsection{Performance Indicators}
Performance Indicators for MOEAs are broadly categorised into four aspects \cite{audet2022}:
\begin{enumerate}
\item\textbf{Cardinality:} Cardinality in this context is the number of non-dominated points that are generated by the MOEA.
\item\textbf{Convergence:} This aspect quantifies how close a set of non-dominated points is from the Pareto front.
\item\textbf{Distribution and spread:} This aspect is divided into two sub-groups: First one is how well-distributed the non-dominated points are on the Pareto front; Second one is the extent of Pareto front approximation and whether the Pareto front contains extreme points. This aspect would account for how diverse the non-dominated points are.
\item\textbf{Convergence and distribution:} This aspect considers both the convergence and distribution of the set of non-dominated points.
\end{enumerate}
There are several performance indicators described in the literature by various authors and their application towards MOPs.

\section{Methodology and experimental design}
\subsection{Problems used}
There are three problems used from the multi-objective test suite in Pymoo - ZDT1, ZDT2, and ZDT3, to perform analysis.
\subsection{MOEAs used}
The two MOEAs used for analysis is Non-dominated Sorting Genetic Algorithm (NSGA-II) and Strength Pareto Evolutionary Algorithm 2 (SPEA2).
\subsection{Performance Indicators}
The three performance indicators used for analysis are IGD (convergence-focussed), Diversity Indicator (distribution-focussed), and hyperarea difference (convergence- and distribution-focussed). The hyperarea difference is considered over hypervolume indicator as the MOPs considered are two-dimensional in nature.

\section{Analysis and results}

\section{Discussion and future work}

\bibliographystyle{ACM-Reference-Format}
\bibliography{references.bib} 



\end{document}
